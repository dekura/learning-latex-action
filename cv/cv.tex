\documentclass{mycv}

\name{Goujin Chen}
\address{\href{https://www.cse.cuhk.edu.hk/}{Department of Computer Science} \\ \href{https://www.cuhk.edu.hk/english/index.html}{The Chinese University of Hong Kong}}
\phone{+86 18986888887}
\email{cgjhaha@qq.com}
\homepage{https://dekura.github.io/}
\github{dekura}
\linkedin{dekura}

\begin{document}

\maketitle%

\section{Research \\ Interests}

I am interested in Machine Learning, EDA, VLSI design. My current focuses include:

\begin{itemize}
  \item Machine Learning in VLSI Design.
  \item Reinforcement learning, computer vision.
\end{itemize}

% ================================================
%                 Education
% ================================================


\section{Education}

\subsection{The Chinese University of Hong Kong}[Hong Kong]
% \begin{positions}
%   \entry{Ph.D. \ in Computer Science}{Nov 2021~--~Feb 2024}
% \end{positions}
% \vspace{-\parskip}%
% \begin{itemize}
%   % \item Ph.D.\ in Computer Science \printdate{Nov 2021~--~Feb 2024}
%   % \item Dissertation: \href{https://repository.hkbu.edu.hk/etd_oa/620}{Authenticated Query Processing in the Cloud}
%   \item Advisor: \href{http://www.cse.cuhk.edu.hk/~byu/}{Prof.~Bei Yu}
% \end{itemize}

% \subsection{The Chinese University of Hong Kong}[Hong Kong]
\begin{positions}
  \entry{M.Sc.\ in Computer Science}{Sep 2019~--~Nov 2020}
\end{positions}
\vspace{-\parskip}%
\begin{itemize}
  \item Advisor: \href{http://www.cse.cuhk.edu.hk/~byu/}{Prof.~Bei Yu}
  % \item  \printdate{}
  % \item Dissertation: \href{https://repository.hkbu.edu.hk/etd_oa/620}{Authenticated Query Processing in the Cloud}
\end{itemize}

\subsection{Huazhong University of Science and Technology}[Wuhan, China]
\begin{positions}
  \entry{Bachelor of Computer Science}{Sep 2015~--~Jun 2019}
\end{positions}
\vspace{-\parskip}%
% \begin{itemize}[label={}]
%   \item  \printdate{}
% \end{itemize}

% ================================================
%                     Works
% ================================================
\section{Relevent \\ Working Experience}

\subsection{\href{https://www.smartmore.com/}{Smartmore Co.Ltd.}}[SHENZHEN, China]
\begin{positions}
  \entry{Research Intern}{Nov 2020~--~Jan 2021}
\end{positions}

% \begin{itemize}
%   \item Responsible for
%   \item Badjs (Developed): An open source big data project developed for detecting, organizing and reporting error information in web pages or mobile applications.
% \end{itemize}


\subsection{\href{https://www.tencent.com/en-us}{Tencent Technology Co.Ltd.}}[SHENZHEN, China]
\begin{positions}
  \entry{Research Intern}{May 2018~--~Nov 2018}
\end{positions}

% \begin{itemize}
%   \item Responsible for web development, including F.E. Development and Web Server Development.
%   \item AI server (Developed): A web server using Node.js to build services for an artificial intelligence online video understanding service. I am the designer and the main developer for it.
%   % \item Badjs (Developed): An open source big data project developed for detecting, organizing and reporting error information in web pages or mobile applications.
% \end{itemize}


% ================================================
%                     Awards
% ================================================

\section{Awards}
\subsection{Scholarship}
\vspace{-\parskip}%
\begin{itemize}
  \item Distinguished Academic Performance Scholarship, CUHK. \printdate{May 2020}
  % \item Entrance Scholarship, CUHK. \printdate{Nov 2019}
  \item National Encouragement Scholarship, HUST, Ministry of Education, PRC \printdate{Nov 2016}
  \item First Class Scholarship, HUST, the highest scholarship in HUST. \printdate{2018, ~2019}
\end{itemize}
\vspace{-\parskip}%
\subsection{Internship}
\vspace{-\parskip}%
\begin{itemize}
  \item First Prize, Tencent SNG Hack Week. \printdate{Jun 2019}
  \item Excellent Intern, Tencent. \printdate{Sep 2019}
\end{itemize}


% ================================================
%                     Projects
% ================================================
\section{Projects}

\begin{description}
  \item[DAMO]: Towards High Accuracy DL-Based OPC With Deep Lithography Simulator. This paper present a novel method for Deep Learning based OPC which results surpass the famous OPC tool Mentor Calibre. The manuscript was accepted by ICCAD2020.
  \item[CUDA-OPC]: This is a CUDA acceleration project that aims to improve the ILT computation efficiency, it speeds up the lithography process nearly 40 times than before.
\end{description}

% ================================================
%                     Skills
% ================================================
\section{Skills}

\begin{description}
  \item[Programming] C/C++, Python, Ruby, Matlab, \LaTeX, Bash, Javascript, Rust, Java
  \item[Machine Learning] Skilled in Pytorch, Tensorflow, and CUDA programming.
  \item[Tools] Vim, Git, macOS, Linux
\end{description}

\section{Publications}%

\publications{mypublications.bib}

\section{Talks}

\begin{enumerate}
  \item \href{https://dekura.github.io/data/slides/20200321-fft.pdf}{CUDA based Convolution and FFT on OPC.} \emph{CUDA Group Presentation.}, CUHK. \printdate{Mar 2020}
  \item \href{https://dekura.github.io/data/slides/20200514-opc.pdf}{DLS-DMO: High Accuracy DL-Based OPC With DLS.} \emph{CUDA Group Presentation.}, CUHK. \printdate{May 2020}
\end{enumerate}

\end{document}



